\begin{frame}
	\frametitle{Pseudo-tested Methods}
	\begin{center}
    \huge{Pseudo-tested Methods}
  \end{center}
\end{frame}

\subsection{Defintion}
\begin{frame}
  \frametitle{Definition}
    \begin{center}
      \textbf{What is a Pseudo-tested Method?}

      \vspace{10mm}
      \includegraphics{images/passing}

      \vspace{10mm}
      \textbf{\textit{Def}}: It will never fail.
    \end{center}
\end{frame}

\subsection{Detection}
\begin{frame}
  \frametitle{Detection}
    \begin{center}
      \textbf{How Can We Detect Pseudo-tested Methods}

			\vspace{20mm}
			\textbf{It is harder than you think!}

    \end{center}
\end{frame}

% \begin{frame}
% \begin{figure}[t!]
% \begin{lstlisting}[language = Python, numbers = left, frame = single, caption = Example of a pseudo-tested method]
% numbers.py:
%   def numberOrder(n):
%     numbersSorted = sorted(n)
%     return numbersSorted
%
%
% test_numbers.py:
%   def test_numbers_ordered():
%     numbers = {1,3,2,4}
%     sortedNumbers = {1,2,3,4}
%     orderedNumbers = numberOrder(numbers)
%     assert numbers == sortedNumbers
% \end{lstlisting}
% \end{figure}
% \end{frame}
